\section{Introduction}
\label{sec:introduction}

In this paper, we will present our project, which is a prototype of the \emph{FeedApp}.
Throughout this paper we will refer to this project as \textbf{poll application}.
Poll application is a full-stack application that allows registered users to create, delete, and vote on polls.
Meanwhile, unregistered users can either choose to register or vote anonymously on existing polls.
In addition to using the core technology stack introduced in this course,
the paper will introduce a new technology into the stack, namely Neo4j.
This will be the new featured technology that will replace Redis.
\\

Throughout this paper we will discuss the functionality and design of the \emph{FeedApp}.
Furthermore, the chosen feature technology will be thoroughly evaluated and compared with Redis,
based on the methods introduced in "\emph{A Framework for Evaluating Software Technology}" by Brown A. W. and Wallnau K. C. \cite{brown:96}
In the latter part of this paper we will provide a brief detailed explanation of how the prototype has been implemented,
with some code snippet for illustration.
In the end of this paper we will evaluate the featured technology, its maturity, quality of documentation and the learning curve to how to use it.
\\



The technology stack implemented in this prototype implementation for this project can be seen in Table~\ref{tab:techstack}.
 \begin{table} [h]
     \centering
        \begin{tabular}{|l|l|l|}
            \hline
            \textbf{Category} & \textbf{Technology} & \textbf{Purpose} \\
            \hline
            Backend & Java/SpringBoot & General platform/framework \\
            \hline
            Web Framework & Spring Web MVC & HTTP/REST API \\
            \hline
            Fronted & React.js & SPA UI \\
            \hline
            Security & Spring Security & Authentication \& Authorization \\
            & JWT & Web Token \\
            \hline
            Database & JPA with Hibernate and H2 & Relational database/ORM, Persistent storage \\
            & Neo4j & NoSQL database, Caching \\
            \hline
            Messaging & RabbitMQ & Event sourcing \\
            \hline
            Deployment & Docker & Containerization \\
            \hline
            Repository & Git & Version control system\\
            \hline
            CI & Github Actions & Software development workflow \\
            \hline
        \end{tabular}
        \caption{The Technology Stack.}
        \label{tab:techstack}
\end{table}

The results from the experiments in this paper when testing out the featured technology, Neo4j,
is obtained using \href{https://jmeter.apache.org/}{JMeter}.
JMeter is an open-source Java-based software application used for testing various application measurements,
e.g.\ latency.
This paper will discuss both the performance and the ease of use between the featured technology and the one it replaces.
\\

The rest of the report is organised as follows: in Section~\ref{sec:design} the functional aspects of the FeedApp application is described.
Section~\ref{sec:technology} consists of the technology assessment of new technology in the project.
In Section~\ref{sec:implementation}, brief details of the prototype's core features and aspects.
Finally, in Section~\ref{sec:conclusion} an overall assessment of the featured technology.

