\documentclass[11pt]{article}


\usepackage{a4wide}
\usepackage{mathptm}
\usepackage{xspace}
\usepackage{amsmath}
\usepackage{graphicx}
\usepackage{algorithm}
\usepackage{algpseudocode}
\usepackage{tikz}
\usepackage{tkz-graph}
\usetikzlibrary{shapes.misc, positioning}
\usepackage{listings}
\usepackage{color}
\usepackage{hyperref}
\usepackage{biblatex}

\definecolor{dkgreen}{rgb}{0,0.6,0}
\definecolor{gray}{rgb}{0.5,0.5,0.5}
\definecolor{mauve}{rgb}{0.58,0,0.82}

\lstset{frame=tb,
  language=Java,
  aboveskip=3mm,
  belowskip=3mm,
  showstringspaces=false,
  columns=flexible,
  basicstyle={\small\ttfamily},
  numbers=left,
  numberstyle=\tiny\color{gray},
  keywordstyle=\color{blue},
  commentstyle=\color{dkgreen},
  stringstyle=\color{mauve},
  breaklines=true,
  breakatwhitespace=true,
  tabsize=3
}
\begin{document}

%----------------------------------------------------------------------------------------
%	TITLE PAGE
%----------------------------------------------------------------------------------------

\newcommand*{\titlePage}{\begingroup % Create the command for including the title page in the document
\centering % Center all text

%----------------------------------------------------------------------------------------
%	TITLE SECTION
%----------------------------------------------------------------------------------------

\vspace*{1in}
{\Huge Project Report} \\ % Title
\vspace{5pt}

{\Large \textsl{}} % Subtitle or further description
\vspace{50pt}

%----------------------------------------------------------------------------------------
%	AUTHOR SECTION
%----------------------------------------------------------------------------------------

\textbf{Isak Yau, \\
		Lily Thi Bui, \\
		Johnny Nhat Trung Le} \\ % Author name

\vspace*{1.3in}

%----------------------------------------------------------------------------------------
%	DESCRIPTION AND DATE SECTION
%----------------------------------------------------------------------------------------

{\Large \textbf{DAT250 Advanced Software Technologies} \\
\vspace{10pt}
Department of Computer Science, \\
Western Norway University of Applied Sciences \\}
\vspace{10pt}
{\large \today} % Month and year published
\vspace*{1in}

%----------------------------------------------------------------------------------------
%	LOGO SECTION
%----------------------------------------------------------------------------------------

\begin{figure}[h!]
	\centering
	\includegraphics[height=70pt]{images/logos/hvl_logo_engelsk.pdf}
\end{figure
\titlePage
\pagebreak

\begin{abstract}

The purpose of this paper is to give an evaluation of different NoSQL technologies in the context of caching ability.
The experiments will compare their performance and their ease of use when integrating it to an existing technology stack.
A variety of methods have been used in order to give an adequate conclusion on different technologies in the context of caching.

\end{abstract}
\pagebreak

%\input{commands}


\section{Introduction}
\label{sec:introduction}

In this paper, we will present our project, which is a prototype of the \emph{FeedApp}.
Throughout this paper we will refer to this project as \textbf{poll application}.
Poll application is a full-stack application that allows registered users to create, delete, and vote on polls.
Meanwhile, unregistered users can either choose to register or vote anonymously on existing polls.
In addition to using the core technology stack introduced in this course,
the paper will introduce a new technology into the stack, namely Neo4j.
This will be the new featured technology that will replace Redis.
\\

Throughout this paper we will discuss the functionality and design of the \emph{FeedApp}.
Furthermore, the chosen feature technology will be thoroughly evaluated and compared with Redis,
based on the methods introduced in "\emph{A Framework for Evaluating Software Technology}" by Brown A. W. and Wallnau K. C. \cite{brown:96}
In the latter part of this paper we will provide a brief detailed explanation of how the prototype has been implemented,
with some code snippet for illustration.
In the end of this paper we will evaluate the featured technology, its maturity, quality of documentation and the learning curve to how to use it.
\\



The technology stack implemented in this prototype implementation for this project can be seen in Table~\ref{tab:techstack}.
 \begin{table} [h]
     \centering
        \begin{tabular}{|l|l|l|}
            \hline
            \textbf{Category} & \textbf{Technology} & \textbf{Purpose} \\
            \hline
            Backend & Java/SpringBoot & General platform/framework \\
            \hline
            Web Framework & Spring Web MVC & HTTP/REST API \\
            \hline
            Fronted & React.js & SPA UI \\
            \hline
            Security & Spring Security & Authentication \& Authorization \\
            & JWT & Web Token \\
            \hline
            Database & JPA with Hibernate and H2 & Relational database/ORM, Persistent storage \\
            & Neo4j & NoSQL database, Caching \\
            \hline
            Messaging & RabbitMQ & Event sourcing \\
            \hline
            Deployment & Docker & Containerization \\
            \hline
            Repository & Git & Version control system\\
            \hline
            CI & Github Actions & Software development workflow \\
            \hline
        \end{tabular}
        \caption{The Technology Stack.}
        \label{tab:techstack}
\end{table}

The results from the experiments in this paper when testing out the featured technology, Neo4j,
is obtained using \href{https://jmeter.apache.org/}{JMeter}.
JMeter is an open-source Java-based software application used for testing various application measurements,
e.g.\ latency.
This paper will discuss both the performance and the ease of use between the featured technology and the one it replaces.
\\

The rest of the report is organised as follows: in Section~\ref{sec:design} the functional aspects of the FeedApp application is described.
Section~\ref{sec:technology} consists of the technology assessment of new technology in the project.
In Section~\ref{sec:implementation}, brief details of the prototype's core features and aspects.
Finally, in Section~\ref{sec:conclusion} an overall assessment of the featured technology.



%\section{Design}
\label{sec:design}

In this section, functional aspects of the Poll application will be described through use cases, domain model and architecture.
The purpose of this section is to get an overview of the functionalities the application offers, and which technologies serves as the foundation of the application.

\subsection{Use cases}

The use case diagram in Figure~\ref{fig:use-case} shows functionalities, possible actors and interface of the application.
The different actors are users which can either have the role as "registered" or "anonymous".
Additionally, the role as "admin" is not included because so far it has no specific functionalities, which distinguish it from registered users.
These actors interact with the application through the interface which is represented as the rectangular box surrounding the functionalities.
Based on this figure, the most important use cases are listed below:

\begin{itemize}
    \item \textbf{Create poll}: Registered users can create a new poll.
    \item \textbf{Delete own poll}: Registered users can delete their own poll either through their dashboard or directly in the polls page, which contains all polls.
    \item \textbf{Vote}: Both registered and anonymous users can vote on a poll.
    \item \textbf{Update vote}: Registered users can change their existing vote by selecting a new vote option and sending the vote.
\end{itemize}

\begin{figure}[H]
	\centering
	\includegraphics[scale=0.35]{figs/use-case-diagram.png}
	\caption{Use case diagram of poll application}
	\label{fig:use-case}
\end{figure}

\subsection{Domain model}
\label{sec:model}

The domain model illustrate important objects and their relations in Figure~\ref{fig:domain_model}.
These objects are required in the poll application because they serve as the fundamental entities, which will be stored in the database.
In addition, their relations will help us understand how objects are connected to each other in order to manage their creations and dependencies.
As shown in the figure, the application has four objects, namely User, Poll, Vote and VoteOption.
It is important to note that other classes and helper classes that are part of business logic, caching or messaging are not included in this model.

\begin{figure}[H]
	\centering
	\includegraphics[scale=0.2]{figs/domain_model.png}
	\caption{Domain model of poll application}
	\label{fig:domain_model}
\end{figure}

For each object, there are some application requirements that must be met. For instance, username, email and password are mandatory when registering a new user.
In addition, email must pass a simple email validation, and username and email must be unique.
All of these fields must be at least three characters. When creating new polls, question, validUntil and creator are required.
Furthermore, validUntil must be a date in the future and there must be at least two vote options.
For vote, userId and voteOption are mandatory, and voteOption must be an option in the poll the user is voting on.
Even though voteOption is not an explicit field in the Vote object, it is required as a way to create an association to which option the user is voting on.
Finally, voteOption will be automatically created when the specific poll is made, and it only requires a caption.

\subsection{Application flow}
The application flow diagram in Figure~\ref{fig:flow_diagram}, shows all the screen transitions on the frontend.
Each screen represents a state, where transitions are represented by arrows and conditions are written in brackets.
The red arrows represent any errors that may occur while interacting with our application, and the green arrows represent the opposite where the errors did not occur.
Furthermore, the black arrows without labels represent how users can navigate to different screens through the navigation bar.

\begin{figure}[H]
	\centering
	\includegraphics[scale=0.6]{figs/flow-diagram.png}
	\caption{Simplified application flow diagram of poll application}
	\label{fig:flow_diagram}
\end{figure}

In this figure, we can see the poll application consists of 5 screens. The \textbf{Main Screen} is the starting point, and connects to \textbf{Register Screen}, \textbf{Login Screen} and \textbf{Polls Screen} via a navigation bar.
For simplicity, all transitions through the navigation bar are not included in the figure. Additionally, in the \textbf{Main Screen}, the button "Get started" sends all users to the \textbf{Register Screen}.
The \textbf{Register Screen} is used to register new users, and a registration may fail if the requirements explained in the \hyperref[sec:domain]{domain model section} are not met.
Users can be directed to the \textbf{Login Screen}, either by clicking on the link "log in" in the register form, or upon successful registration when the user will be redirected automatically.
The \textbf{Login Screen} has similar behaviour as the \textbf{Register Screen}, but with different constraints such as the user needs to be registered in order to log in.
When users successfully log in, they will be redirected to their personal dashboard. To access the \textbf{Dashboard Screen}, all users must be logged-in.
This requirement also applies to creating new polls. The user can also delete and vote on their own poll from the dashboard.
However, a registered user can only vote one time on a poll, but may change their existing vote by choosing another option.
To log out, the logged-in user can click on the "Log out" button in the navigation bar. Finally, the \textbf{Polls Screen} contains all created polls where both registered and anonymous users can vote.
The vote button is disabled if the user has not chosen an option. This means pressing a disabled button will do nothing. Registered user can also delete their own polls from this screen.

\subsection{User Screens}
The user screen mock-ups were made with Figma (https://www.figma.com/) and illustrates the initial design of the poll application.
Some prototypes are displayed below:

\begin{figure}[H]
	\centering
	\includegraphics[scale=0.22]{figs/register-screen.png}
	\caption{The Register Screen}
	\label{fig:register}
\end{figure}

In the \textbf{Register Screen}, users can either register as new users or be forwarded to the login screen if they click on the "Log in" link.
When pressing the "Register" button, all data fields in the form will be checked against the explained requirements in the \hyperref[sec:domain]{domain model section}.
All error messages are red and tailored for the specific data field. For example, one error message is "Username must be longer than 3 characters".

\begin{figure}[H]
	\centering
	\includegraphics[scale=0.22]{figs/dashboard-screen.png}
	\caption{The Dashboard Screen where user can create new polls}
	\label{fig:dashboard}
\end{figure}

In the \textbf{Dashboard Screen}, users can view their own polls, delete and create new ones. To create a new poll the users click on the button "Create poll", which will show a modal with the poll form.
Creating a new poll will only be successful if they meet the requirements. These requirements will automatically be checked when clicking the "Create" button.
Deleting a poll is done by clicking on the red button "Delete".

\begin{figure}[H]
	\centering
	\includegraphics[scale=0.22]{figs/polls-screen.png}
	\caption{The Polls Screen}
	\label{fig:register}
\end{figure}

In the \textbf{Polls Screen}, both anonymous and registered users can vote on polls. To vote on a poll, users need to choose a vote option and click the "Vote" button.
A vote option is chosen if it gets a purple border color. If no option is chosen, the button will remain disabled which is identified by a grey color.
After the users have voted, the results for the specific poll will be updated. The total votes on the poll is displayed, while the progress bar indicate the number of votes on each vote option.

\subsection{Architecture}
Finally, Figure~\ref{fig:architecture} shows the system architecture diagram, which highlights what components exist in the poll application and what technologies are used to implement them.

\begin{figure}[H]
	\centering
	\includegraphics[scale=0.5]{figs/system_architecture.png}
	\caption{System architecture of poll application}
	\label{fig:architecture}
\end{figure}

The backend uses Spring Web MVC to build the REST API, which provides several endpoints that can be requested by the frontend.
Service classes provide the business logic which enforce the rules to determine how data is managed.
RabbitMQ is the technology used for messaging, while persistent storage is handled by Java Persistence API with H2.
Finally, Neo4j is used for caching and the frontend is built with React.


%

%\section{Technology Assessment}
\label{sec:technology}


In this section, the feature technology of the project will be presented and compared to another technology in the context of caching.
As mentioned in the introduction, the chosen feature technology is Neo4j which will replace Redis.
These technologies are different types of NoSQL databases which are optimized for certain use cases.
Redis is a hybrid database and caching system which stores data as key-value pairs in-memory.
This makes it a common choice for caching.
On the other hand, Neo4j is a graph-based database which stores data in nodes and relationships.
Therefore, this paper will explore and compare their capabilities in terms of caching ability by following the evaluation framework from \cite{brown:96}, see Figure~\ref{fig:framework}.

\begin{figure}[thb]
    \centering
    \includegraphics[scale=0.6]{figs/framework.png}
    \caption{Software technology evaluation framework.}
    \label{fig:framework}
\end{figure}

\subsection{Descriptive Modeling}

To get a better understanding of Neo4j and Redis,
their genealogy is illustrated in Figure~\ref{fig:genealogy}.

The genealogy shows that Neo4j and Redis are different types of NoSQL databases.
In particular, Neo4j is a graph database where entities are represented as nodes,
and relationships between entities are represented as edges.
Graph databases have existed for a long time, but have not been commercially used until the mid-to-late 2000s \cite{graphDB}.
During this time the first version of Neo4j was released in February 2010 and has since been gaining popularity \cite{neo4jWiki}.
As of November 2025, Neo4j is the most popular graph database according to DB-Engines \cite{graphDBRank}.
Similar to other types of graph databases, Neo4j uses its own query language called Cypher.
Cypher constructs are declarative, and have similar clauses found in SQL \cite{cypherNeo4j}.
The advantage of using graph databases is clear when it comes to storing complex relationships,
as these actions can be handled directly.
\\

\begin{figure}[H]
    \centering
    \includegraphics[scale=0.5]{figs/genealogy.png}
    \caption{Neo4j/Redis genealogy.}
    \label{fig:genealogy}
\end{figure}

Additionally, looking up relationships can be computationally expensive in certain scenarios \cite{Neo4j}.
This is because the join queries in relational databases can grow complex and nested as the relationships get advanced.
Thus, storing data relationships directly is beneficial in scenarios such as network management system,recommendation system and fraud detection.
In those scenarios, having a quick and efficient traversal between data relationships is beneficial.
Hence, graph databases are ideal for projects focused on the relationship between data rather than the data itself \cite{Neo4jGettingStarted}.
Since graph databases do not use a predefined schema, they are highly flexible.
Other graph based databases which share similar functionalities as Neo4j are shown as “peers” in Figure~\ref{fig:genealogy}.
\\

On the other hand, Redis is both a NoSQL database and a hybrid/cache database which makes it suitable for a variety of different use cases.
For instance, the most common use case of Redis is to use it as a cache.
However, it is possible to use Redis for durable storage \cite{RedisP}.
Redis was created in 2009 and the motivation was to fix the poor scalability in the founder’s startup.
Redis has seen massive growth since its release, achieving their record growth in 2025 \cite{RedisWiki}.
Similarly to its “peers”, Memcached, Riak and Etcd, Redis stores data as key-value pairs.
\\

An alternative way of showcasing the use cases and competitors of Redis and Neo4j, is the problem domain habitat in Figure~\ref{fig:habitat}.

\begin{figure}[thb]
    \centering
    \includegraphics[scale=0.5]{figs/habitat.png}
    \caption{Elements of the Neo4j/Redis habitat.}
    \label{fig:habitat}
\end{figure}

This figure illustrates the technologies, their respective competitors and the problem context.
A problem context is a class of problems addressed by the technology \cite{brown:96}.
Whilst, the special characteristic represents the technology’s ability to solve the problem context.

\subsection{Experiment Design}

In this section, we will plan how we will evaluate and compare the technologies,
where the overall goal is to evaluate how different choices of technology can affect the caching ability.
The purpose of caching is to improve performance by storing data in a temporary storage,
effectively reducing the need for retrieving data from a permanent database \cite{awsCache}.
Therefore, it is particularly interesting to compare two NoSQL databases as they have different ways of storing data due to their underlying architecture and structure.
\\

In order to effectively evaluate these technologies in the predefined context,
the following hypotheses have been made:


\begin{enumerate}
    \item As Redis stores data through key-value pairs and Neo4j stores data in graphs, Redis is faster in terms of adding and retrieving data.
\end{enumerate}

Based on the results of the mentioned hypotheses above, the last hypothesis is:

\begin{enumerate}
    \item[2.] In terms of usability and complexity, Redis is simpler to implement than Neo4j.
\end{enumerate}

The paper tries to answer these hypotheses by conducting an experiment which is a simplified version of the final project.
In order to do so, a branch was made for each technology based on the main branch.
Then each technology was implemented in a way that serves as a caching service for the main application.
This is to get a better understanding of its overall fit with the existing technology stack.
\\

The experiment can be divided into two parts.
First, simulated users will vote on a specific poll where each vote will be stored according to the chosen technology.
Then, handlers will retrieve all votes which triggers the caching logic.
In this part of the experiment, we can verify that the correct number of votes are saved to the cache.
Furthermore, we can confirm that the data is retrieved from the cache and not the permanent database in the correct setting.
\\

To ensure consistency in both parts of the experiment, it was decided to use JMeter as the tool to perform a load test.
This experiment would help refute or support the proposed hypotheses.
Since the first hypothesis is related to data retrieval, the total number of simulated users is 1000.
It is important to note that this single experiment will address all of the hypotheses, as the setup and experiment design would be similar for each.
In addition, the experiment includes the same requests and parameters for each technology to ensure a fair comparison.
\\

\subsection{Experiment Evaluation}

For the following section we will discuss the main findings from our experiment explained in the section above.
\\


\begin{figure}[H]
    \centering
    \hspace*{-2cm}\includegraphics[scale=0.4]{figs/results}
    \caption{ Result overview of Neo4j/Redis experiment.}
    \label{fig:results}
\end{figure}

\noindent\textbf{H1 - As Redis stores data through key-value pairs and Neo4j stores data in graphs,
    Redis is faster in terms of adding and retrieving data}
\\

After conducting the experiment for Neo4j and Redis,
JMeter provided us with a general statistic table which can be seen in Figure~\ref{fig:results}.
The results from the table included several different metrics.
An interesting metric to look at is the average response time, which indicates how fast a request could be satisfied by the server.
From the results, it is clear that Redis outperforms Neo4j in that metric by a significant margin with a difference of 6.77 ms.
Redis with an average of 2.64 ms, and Neo4j with 9.41 ms.
Furthermore, another metric that piqued our interest was the Max-metric where Neo4j has its highest response time at 87 ms,
whilst Redis has its highest peak at 55 ms.
To our knowledge, the high peaks are caused by the sets of initial requests,
which have the responsibility to establish a connection with the application.
\\

Observe that POST requests take longer time to process than GET requests for both technologies.
From our understanding, this can be due to the fact that creating objects requires additional business logic and validations.
\\

Other metrics such as Throughput and Network are not relevant metrics for our evaluation,
as they are not directly affected by the chosen technologies.
After all, both technologies perform the same simulated load test with the same parameters.
Hence, Throughput and Network gives similar results for Neo4j and Redis.
\\

A factor that might explain why Neo4j has a higher response time than Redis could be their difference in architecture.
Storing data in a key-value data structure could be less computationally expensive than constructing a graph to represent the data.
\\

The highlighted observations support the first hypothesis that Redis is quicker than Neo4j both in terms of adding and retrieving data to the cache.
Therefore, we have reasons to accept our first hypothesis, as the results support this claim.
\\

\noindent\textbf{H2 - In terms of usability and complexity, Redis is simpler to implement than Neo4j }
\\

When it comes to the setup, Redis was easier and more understandable compared to Neo4j.
In both technologies, we struggled with integrating it to our application.
We expected that it would be easier to work with Jedis (Java Redis), as we had some prior experience.
However, we faced challenges with integrating it to our application.
In retrospect, this negative experience should not affect the use of Redis as a cache, because we realized that most of the issues stem from our existing code.
Navigating through the official documentation to look at what functionalities Redis offers was a bit challenging.
Although the official documentation was structured in a well-organized manner, it still took some time to find what we were after.
\\

As for Neo4j, we encountered similar issues with the implementation.
After struggling to properly integrate it in our code, two issues were found when we simulated the load test.
\\

Firstly, due to the nature of NoSQL being schemaless, it enabled the possibility for multiple nodes having the same ID.
Because of our lack of knowledge and experience working with Neo4j, we were confused as to why our unique ID constraint were not enforced in the database.
This resulted in our load tests failing at multiple instances, making us unsure where the core issue actually was.
To resolve this issue we had to manually add the constraint for unique ID to the Neo4j database.
\\

Secondly, Spring Boot’s Neo4j repository did not function as we expected it to do.
From our understanding, we suspect it was because the data types from the database would not map properly to the appropriate Java types.
Since our experience with Spring Boot repositories have been working consistently without any problems, we did not suspect that the culprit for the issue was the Neo4j repository itself.
Our solution was to create a custom service class, and use the Neo4jClient application to manually bind the return value to our desired Java type.
\\

Overall, both Redis and Neo4j had the same struggles with integrating to our code base.
However, we spent much more time implementing Neo4j compared to Redis, due to our inexperience with Neo4j.
In addition, by comparing lines of code, Neo4j required more than Redis.
To conclude, we accept the hypothesis that Redis is simpler than Neo4j, both in terms of usability and complexity.
\\

\noindent\textbf{Limitations}
\\

Nevertheless, it is important to report significant limitations to our experiment.
Firstly, our implementation of Neo4j as a cache did not support time to live (TTL).
Neo4j by default does not support any direct TTL functionality unlike Redis.
A possible solution would be to create a special query that would clear up stale data given a timestamp.
This would require our application to have an internal clock that would execute this query periodically.
However, in our application, we were unsuccessful with creating such a solution.
Without a way to automatically clear stale data, it hinders the efficiency of using Neo4j as a cache.
In other words, Neo4j is more suitable as a long term persistent storage, which defeats the purpose of what a cache represents.
\\

Secondly, our implementation of Redis is using in-memory storage instead of disk like Neo4j.
This gives an unfair advantage to Redis, because reading from in-memory will always be faster than disk reading.
There was an effort in trying to accomplish persistence mode in Redis, but due to time constraint and lack of knowledge,
we were unable to properly integrate disk storing.
Therefore, these results may not be an accurate representation of how well Neo4j would perform as a cache in comparison with Redis.
However, we believe that the results would be the same even if Redis was using disk storage instead of in-memory storage.
As mentioned earlier, this can be because of the architecture of Redis compared to Neo4j.
Based on the results of our two hypotheses, we conclude that Redis is more suitable as a cache than Neo4j.
\\









%
%\section{Prototype Implementation}
\label{sec:implementation}
\subsection{Prototype Overview}

As mentioned in the \hyperref[sec:introduction] {introduction},
the purpose of this prototype was to develop a poll application with core features such as
user registration, voting and poll creation.
The prototype also integrates a caching mechanism which serves as the base for experimenting with
the feature technology in this research project.
The following core features that can be found in this prototype are listed below:

\begin{itemize}
    \item \textbf{User registration}: Users can register an account to create and publish polls.
    \item \textbf{Voting}: Each poll has at least two voting options, and users can cast their votes either as registered users or as anonymous users.
    \item \textbf{Vote update}: A registered user that voted on a poll can change their vote, while an anonymous voter cannot.
\end{itemize}
\\
A feature within the poll application is
that users can \textbf{register}, \textbf{login} and \textbf{logout} via Spring Security.
Authentication is implemented using \textbf{JWT}, while authorization uses \textbf{Role-Based Access Control}.
For data storage, the poll application utilizes an \textbf{H2 in-memory database} using \textbf{JPA with Hibernate},
while the cache stores voting results for efficient access.
\\

Additionally, \textbf{RabbitMQ} is integrated to log vote events in the application.
The idea behind messaging was to allow creators of their own polls to be notified
whenever a vote had been cast to their poll.
However, due to time constraints and the way functionalities were prioritized, this was never implemented in the prototype.
Thus, future developers can implement endpoints that can push those events to the client-side, where the frontend can filter out
the relevant vote events for the signed inn user.
Other functionalities that can be improved within this poll application is to add role functionalities that are unique for
ADMIN users and registered USERS. Within the poll application, roles are implemented, but they are not distinguished in a manner where ADMIN has functionalities
that separate them from registered USERS.

\subsection{Run application}
There are two ways to run this application.
The simplest way is to use Docker containers to run this application.
The poll application has already been packaged with all necessary dependencies such as RabbitMQ and Neo4j, using a Docker Compose setup.
That way, the poll application and its required service can be launched together in a consistent and isolated environment.
All it requires is to open a terminal within the root of the project, and run the command
\begin{quote}
    \texttt{docker-compose up --build}
\end{quote} on the terminal.
It will automatically run containers for each of the dependencies and the poll application.
Once they are all running, the poll application can be accessed at localhost:8080.
The other way would be to download all the software dependencies, which require more manual steps.
Users have to install RabbitMQ and Neo4j in their host system where they are connected to the following ports listed below:
\begin{itemize}
    \item \textbf{RabbitMQ}: default client port 15672:15672 and default UI port 5672:5672.
    \item \textbf{Neo4j}: default client port 7687:7687 and default UI port 7474:7474.
\end{itemize}

Once all software dependencies have been mapped to the appropriate ports and are running,
the application can be started with the following two ways.
\\
\\
For Linux or macOS:
\begin{quote}
\texttt{./gradle bootRun}
\end{quote}
\\
For Windows OS:
\begin{quote}
\texttt{.\textbackslash gradlew.bat bootRun}
\end{quote}
\\
The other way is to go to the class \texttt{FeedApplication} located at
\begin{quote}
\texttt{backend/src/main/java path from the root}
\end{quote}
where the user can start it by running the main function:
\lstinputlisting[language=java]{code/FeedAppApplication.java}
%
%\section{Conclusions}
\label{sec:conclusion}


We conclude that Redis is a better choice than Neo4j when it comes to caching.
Redis has a better performance than Neo4j in terms of speed.
This is supported by the results from our experiment,
indicating that the average response time in Redis is significantly lower than Neo4j.
When it comes to ease of use,
our experience from implementing both technologies concludes that Redis is much simpler to work with than Neo4j.
Even though we struggled with implementing both technologies into our application,
Neo4j had more severe issues than Redis.
And the issues were not easy to pinpoint, due to the lack of experience and knowledge with using Neo4j.
On the other hand, Redis was something we have encountered before, giving us some familiarity with the technology.
Despite these challenges, there was a steep learning curve getting to know a completely new technology.
Not only did we dive into an unfamiliar database structure, but we also learned a new query language.
Overall, after finishing this project,
we have a greater understanding of how to evaluate different technologies to determine their relevance and efficiency.
In addition, we extend our knowledge by adding another technology to our tool box.



\bibliographystyle{plain}
%\bibliography{report.bib}
%\printbibliography

\end{document}
